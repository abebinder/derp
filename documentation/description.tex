\chapter*{Project Description}
Delivery of projects on time with sufficient functionality is a regular problem for development teams. Despite a plethora of project management techniques and practices in industry, the majority of projects remain overbudget, lacking functionality, and late. The OSNAP project management office (PMO) has observed that a major contributor to project failure is a lack of visibility and communication between the development organization and the rest of the business. These communication failures involve insufficient contact, development is highly productive but builds a solution that does not match with the actual requirements, and overwhelming inefficient contact, development is unable to make forward progress due to interruptions and direction changes coming from frequent meetings.

To address these challenges, the PMO has suggested that project management adopt daily standup meetings and weekly contact with the business to verify that development and business needs are tracking together. The process has been effective at increasing project success however the distributed and part time nature of most OSNAP developers has created problems for standup meeting scheduling. Additionally, a large amount of developer time wasted on standup meetings for the larger development teams. Implementation of a software platform to handle the reporting information could save OSNAP thousands of hours per year in time lost to the existing process. The Development Experience Reporting Platform (DERP) will allow OSNAP to realize these savings.

The major cost for the existing process is in the daily standup meetings. DERP's daily report will replace the daily standup meeting. Each day, whenever it is convenient for the developer, the developer will submit a short status via DERP. Each morning at 6am, DERP will send a report of the status messages received for the previous day to the responsible project manager. The project manager will respond directly to individual developers as needed based on the daily status message via email or other mechanism outside of DERP.

Weekly reporting has been an extremely effective method for messaging between the business and development teams on previous projects. Additionally, the history provided by weekly status reporting provides a solid basis for project postmortem meetings. Moving weekly status to DERM will avoid the occasionally lost messages from the current email based process. Like daily status, weekly status reports can be submitted at times convenient for the developers. Weekly status reports will conform to the OSNAP PMO standards and will contain accomplishments, next steps, and upcoming challenges. A weekly status report may also contain general comments. Project managers will be able to view weekly reports through the DERP interface.